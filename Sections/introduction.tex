Characterization of single photon avalanche diodes (SPAD) is a practice that consists on experimentaly quantifying its four main performance indicators known as: 
jitter, photon detection probability (PDP), dark count rate (DCR) and deadtime. 
This is done with multiple techiques such as retroluminecence \cite{}, microscopic techniques \cite{}, and Time-Correlated single photon counting (TCSPC). \\

This Lab-Report is focused on measuring SPAD's performance indicators through a series of TCSPC experiments using different light sources. 
The goal is to report the temporal response of integrated SPAD detectors together with the statistical properties of the light sources. 

\subsection{Timing SPAD Detector Response Function: Jitter}
A photon incident of the active area of a SPAD detector, triggers an avalanche of charge carriers creating a current signal. 
The current is later detected and converted to a digital signal; typicaly within less than $100 ps$.
This temporal delay between the time of photon arrival and signal detection is called jitter. \\

Timing jitter is quantified by measuring delay $\Delta t$ between the emission of 
a LASER pulse (recorded by a Laser Clock), and the detection signal from the SPAD. 
Accumulating these events in a histogram it is possible to extract the instrument response function (IRF). 
The jitter is then reported as the Full Width at Half Maximum (FWHM) of this distribution. 

\subsection*{Detection Probabilities (Pulsed Source)}
A pulsed LASER can be described by its repetition rate $f_{rep}$.
Thus, a total number of pulses $N_{pulses}$ in an exposure time $t_{exp}$, is given by $N_{pulses} =f_{rep} \times t_{exp} $. 
If a SPAD registers $C_A$ total counts during this exposure, the average probability of detecting a photon per pulse is:
\begin{equation}
    P_A = \frac{C_A}{N_{pulses}} = \frac{R_A}{f_{rep}}
\end{equation}
where $R_A = C_A / T_{exp}$ is the average count rate in counts per second (cps). 
$P_A$ also known as Photon Detection Probability (PDP), is typically $< 0.05$.

\subsection*{Statistical Properties of Light}

\subsubsection*{Second Order Coherence $g^{(2)}$}
The second order correlation function $g^{(2)}$ quantifies intensity correlations, photon statistics and time separated probability of detecting two photons. 
A value for $g^{(2)} < 1$, is an indicator of antibuching; a quantum characteristic of the respective light source. \\

Experimentally -- for a pulsed LASER set-up, $g^{(2)}$ can be calculated using:
\begin{equation}
    g^{(2)}(0) = \frac{P_{AB}}{P_A P_B} = \frac{C_{AB} \cdot f_{rep}}{R_A R_B T_{exp}}
\end{equation}
For an ideal coherent state (Poissonian statistics), $g^{(2)}(0) = 1$.\\

For the CW LASER and rotating diffuser setup, the delay $\tau$ is continuous. The normalized coincidence rate across a bin width $\Delta \tau$ is:
$$ g^{(2)}(\tau) = \frac{C_{AB}(\tau)}{R_A R_B \Delta \tau T_{exp}} $$

True thermal light exhibits photon bunching. As the delay approaches zero, the statistics shift from Poissonian to Bose-Einstein, yielding a theoretical limit of:
$$ g^{(2)}(0) = 2 $$
The decay of $g^{(2)}(\tau)$ back to 1 as $\tau$ increases maps the coherence time ($\tau_c$) of the pseudothermal source, which is strictly governed by the rotational speed of the optical diffuser.

%This is the introduction. Describe the background and motivation for the work here. 
%This is an example of a citation \cite{einstein1905}.
